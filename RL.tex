\documentclass[12pt]{article}

% 数学宏包
\usepackage{amsmath, amssymb, amsthm}
\usepackage{CJK}
\usepackage[ruled,vlined]{algorithm2e}
\usepackage{booktabs}   % 提供 \toprule, \midrule, \bottomrule 等命令,适合学术表格
\usepackage{array}
\usepackage{float}      % 提供 [H] 浮动控制选项,让表格固定在当前位置
\usepackage{caption}    % 改进表格标题样式
\usepackage{xcolor}     % 如果你想为表格添加轻微底色或高亮
\usepackage{graphicx}

% 定理环境设置
\theoremstyle{definition} % 定义类(直立字体)
\newtheorem{definition}{Definition}[section]

\theoremstyle{plain} % 定理类(斜体)
\newtheorem{theorem}[definition]{Theorem}
\newtheorem{lemma}[definition]{Lemma}
\newtheorem{corollary}[definition]{Corollary}
\newtheorem{proposition}[definition]{Proposition}

\theoremstyle{remark} % 备注类(直立字体,带斜体标题)
\newtheorem{remark}[definition]{Remark}
\newtheorem{example}[definition]{Example}

% 证明环境
\renewcommand\qedsymbol{$\blacksquare$} % QED 符号改成实心方块

% 页面与字体
\usepackage[a4paper, margin=1in]{geometry}
\usepackage{lmodern} % 更好看的字体

% 数学集合常用符号
\newcommand{\R}{\mathbb{R}}
\newcommand{\Z}{\mathbb{Z}}
\newcommand{\N}{\mathbb{N}}

\begin{document}
\begin{CJK}{UTF8}{gbsn}

\section{Intro}
\section{Imitation learning}
\begin{flushleft}
\begin{tabular}{ll}
$\mathbf{s}_t$ & state \\
$\mathbf{o}_t$ & observation \\
$\mathbf{a}_t$ & action \\
$\pi_\theta(\mathbf{a}_t \mid \mathbf{o}_t)$ & policy \\
$\pi_\theta(\mathbf{a}_t \mid \mathbf{s}_t)$ & policy (fully observed) \\
\end{tabular}
\end{flushleft}
In regular supervised learning, we have the i.i.d assumption, i.e. training points does not affect each other. However, we don't have such assumption in RL, taking actions affects
next observations.

\section{Policy Gradient Methods}

\subsection{Motivation}

Value-based methods (e.g., Q-learning) estimate action values and then derive the policy indirectly by taking the action with the highest value.  
In contrast, \textbf{policy gradient methods} directly optimize the parameters of a policy
\(\pi_\theta(a|s)\) with respect to expected return.  
They are especially suitable for:
\begin{itemize}
    \item Continuous or high-dimensional action spaces
    \item Stochastic policies (exploration via inherent randomness)
\end{itemize}

---

\subsection{Problem Setup}

Let the agent follow a parameterized policy \(\pi_\theta(a|s)\).
The goal is to maximize the expected discounted return:

\[
J(\theta) = \mathbb{E}_{\tau \sim \pi_\theta} [ R(\tau) ]
\]

where a trajectory \(\tau = (s_0,a_0,s_1,a_1,\ldots)\) is generated according to:
\[
p(\tau|\theta) = \rho_0(s_0) \prod_{t=0}^{T-1} \pi_\theta(a_t|s_t) P(s_{t+1}|s_t,a_t),
\]
and the cumulative discounted reward is
\[
R(\tau) = \sum_{t=0}^{T-1} \gamma^t r(s_t,a_t).
\]

---

\subsection{The Policy Gradient Theorem}

We aim to compute the gradient of \(J(\theta)\) with respect to \(\theta\):
\[
\nabla_\theta J(\theta) = \nabla_\theta \mathbb{E}_{\tau \sim \pi_\theta}[R(\tau)].
\]

Expanding the expectation:
\[
\nabla_\theta J(\theta)
= \int R(\tau) \nabla_\theta p(\tau|\theta) d\tau.
\]

Applying the \textbf{log-derivative trick}:
\[
\nabla_\theta p(\tau|\theta) = p(\tau|\theta)\nabla_\theta \log p(\tau|\theta),
\]
we obtain:
\[
\nabla_\theta J(\theta)
= \mathbb{E}_{\tau \sim \pi_\theta} \left[
R(\tau)\, \nabla_\theta \log p(\tau|\theta)
\right].
\]

Now note that:
\[
\log p(\tau|\theta) = \sum_{t=0}^{T-1} \log \pi_\theta(a_t|s_t),
\]
so:
\[
\nabla_\theta J(\theta)
= \mathbb{E}_{\tau \sim \pi_\theta} \left[
\sum_{t=0}^{T-1} R(\tau)\, \nabla_\theta \log \pi_\theta(a_t|s_t)
\right].
\]

This is the foundation of the \textbf{REINFORCE algorithm}.

---

\subsection{Reducing Variance with Return-to-Go}

Instead of using the total return \(R(\tau)\) for each timestep,
we can use the \emph{return-to-go} \(G_t\):

\[
G_t = \sum_{k=t}^{T-1} \gamma^{k-t} r(s_k,a_k)
\]

Then, the gradient estimator becomes:
\[
\nabla_\theta J(\theta)
= \mathbb{E}_{\tau \sim \pi_\theta} \left[
\sum_{t=0}^{T-1} G_t\, \nabla_\theta \log \pi_\theta(a_t|s_t)
\right].
\]

This reduces variance since it only attributes future rewards to current actions.

---

\subsection{Baseline and Advantage Function}

To further reduce variance, we subtract a baseline \(b(s_t)\) that does not depend on the action:
\[
\nabla_\theta J(\theta)
= \mathbb{E}_{\tau \sim \pi_\theta} \left[
\sum_{t=0}^{T-1} (G_t - b(s_t))\, \nabla_\theta \log \pi_\theta(a_t|s_t)
\right].
\]

A common and effective choice is:
\[
b(s_t) = V^\pi(s_t)
\]

Then, the term \(G_t - V^\pi(s_t)\) is the \textbf{advantage function}:
\[
A^\pi(s_t,a_t) = Q^\pi(s_t,a_t) - V^\pi(s_t)
\]

Thus, the general policy gradient becomes:
\[
\boxed{
\nabla_\theta J(\theta)
= \mathbb{E}_{s_t,a_t \sim \pi_\theta}
\left[ A^\pi(s_t,a_t)\, \nabla_\theta \log \pi_\theta(a_t|s_t) \right].
}
\]

---

\subsection{Actor–Critic Methods}

In practice, we approximate:
\begin{itemize}
    \item \(A^\pi(s,a)\) or \(V^\pi(s)\) using a critic network (value function estimator)
    \item \(\pi_\theta(a|s)\) using an actor network (policy)
\end{itemize}

The critic is trained by minimizing:
\[
L_V(\phi) = \frac{1}{2}\, \mathbb{E}_{s_t \sim \pi_\theta}
\left[ (V_\phi(s_t) - G_t)^2 \right].
\]

The actor is updated via:
\[
\nabla_\theta J(\theta)
\approx \mathbb{E}_{s_t,a_t \sim \pi_\theta}
\left[ (G_t - V_\phi(s_t))\, \nabla_\theta \log \pi_\theta(a_t|s_t) \right].
\]

---

\subsection{Intuitive Summary}

\begin{itemize}
    \item The policy gradient optimizes expected return directly.
    \item The gradient direction is determined by:
    \[
    \nabla_\theta \log \pi_\theta(a_t|s_t)
    \]
    scaled by how good the action was:
    \[
    A^\pi(s_t,a_t)
    \]
    \item The baseline reduces variance without introducing bias.
    \item Actor-Critic = policy gradient + value function approximation.
\end{itemize}

---

\subsection{Algorithm: REINFORCE}

\begin{algorithm}[H]
\caption{REINFORCE Algorithm}
Initialize policy parameters $\theta$ \;
\Repeat{}{
    Collect trajectory $\tau = (s_0,a_0,r_0,\dots,s_T)$ by running $\pi_\theta$ \;
    \For{each time step $t = 0, \dots, T-1$}{
        Compute return-to-go $G_t = \sum_{k=t}^{T-1} \gamma^{k-t} r_k$ \;
        Update policy: $\theta \leftarrow \theta + \alpha G_t \nabla_\theta \log \pi_\theta(a_t|s_t)$ \;
    }
}
\end{algorithm}



\section{Fitted Q-Iteration, Function Approximation, and Bellman Operator}

\subsection{Function Approximators}

In reinforcement learning, we aim to estimate the optimal action-value function \( Q^*(s,a) \).  
When the state-action space is large or continuous, we cannot store one value per pair, so we approximate it with a parameterized function \( Q_\phi(s,a) \):

\[
Q_\phi(s,a) \approx Q^*(s,a)
\]

Common types of function approximators include:

\begin{itemize}
    \item \textbf{Tabular:} Each \( (s,a) \) pair has a separate entry.
    \item \textbf{Linear:} \( Q_\phi(s,a) = \phi^\top x(s,a) \)
    \item \textbf{Nonlinear:} Neural networks \( Q_\phi(s,a) = f_\phi(s,a) \)
    \item \textbf{Kernel / Gaussian process:} \( Q_\phi(s,a) = \sum_i \alpha_i k((s,a),(s_i,a_i)) \)
\end{itemize}

---

\subsection{The Bellman Optimality Operator}

The Bellman optimality equation is defined as:

\[
Q^*(s,a) = r(s,a) + \gamma \mathbb{E}_{s' \sim P(\cdot|s,a)}\left[\max_{a'} Q^*(s', a')\right]
\]

Define the \textbf{Bellman optimality operator} \( \mathcal{T}^* \) acting on any function \( Q \):

\[
(\mathcal{T}^* Q)(s,a) = r(s,a) + \gamma \mathbb{E}_{s'}\left[\max_{a'} Q(s',a')\right]
\]

Then the optimal \( Q^* \) satisfies the fixed-point condition:

\[
Q^* = \mathcal{T}^* Q^*
\]

That is, \( Q^* \) is the fixed point of the Bellman operator.

---

\subsection{Contraction Property of the Bellman Operator}

The Bellman operator \( \mathcal{T}^* \) is a \(\gamma\)-\textbf{contraction mapping} under the sup norm \( \|\cdot\|_\infty \):

\[
\|\mathcal{T}^* Q_1 - \mathcal{T}^* Q_2\|_\infty \le \gamma \|Q_1 - Q_2\|_\infty, \quad \forall Q_1, Q_2
\]

This implies that repeated application of \( \mathcal{T}^* \) moves any \( Q \) closer to \( Q^* \).

---

\subsection{Banach Fixed-Point Theorem}

\begin{theorem}[Banach Fixed-Point Theorem]
Let \( (X, d) \) be a complete metric space and \( T : X \to X \) be a contraction mapping, i.e.
\[
d(Tx, Ty) \le \gamma d(x,y), \quad 0 \le \gamma < 1.
\]
Then:
\begin{enumerate}
    \item \( T \) has a unique fixed point \( x^* \in X \).
    \item Starting from any \( x_0 \in X \), the sequence \( x_{k+1} = T(x_k) \) converges to \( x^* \).
\end{enumerate}
\end{theorem}

Applying this theorem with \( X = \mathbb{R}^{|S||A|} \), \( d = \|\cdot\|_\infty \), and \( T = \mathcal{T}^* \),
we conclude that \( Q^* \) is the unique fixed point and
\[
Q_{k+1} = \mathcal{T}^* Q_k \quad \Longrightarrow \quad Q_k \to Q^*.
\]

---

\subsection{Function Approximation and the Projected Bellman Operator}

When using function approximators such as neural networks, we can only represent a subset of all possible functions.  
Hence, we define a \textbf{projected Bellman operator}:
\[
\mathcal{T}_\Pi Q = \Pi \mathcal{T}^* Q,
\]
where \( \Pi \) denotes projection onto the function space representable by \( Q_\phi \).

However, \( \mathcal{T}_\Pi \) is generally \emph{not} a contraction mapping:
\[
\|\mathcal{T}_\Pi Q_1 - \mathcal{T}_\Pi Q_2\|_\infty \not\le \gamma \|Q_1 - Q_2\|_\infty,
\]
so convergence is not guaranteed. This explains why nonlinear function approximators such as deep neural networks may cause instability or divergence.

---

\subsection{Summary Table}

\begin{table}[h!]
\centering
\begin{tabular}{lccc}
\hline
\textbf{Setting} & \textbf{Q Representation} & \textbf{Bellman Operator Type} & \textbf{Convergence Guarantee} \\
\hline
Tabular & Lookup table & Exact \( \mathcal{T}^* \) & Guaranteed (Banach theorem) \\
Linear Approximation & \( Q_\phi(s,a) = \phi^\top x(s,a) \) & Approximate & Sometimes guaranteed \\
Neural Networks & \( Q_\phi(s,a) = f_\phi(s,a) \) & Projected \( \mathcal{T}_\Pi \) & No theoretical guarantee \\
\hline
\end{tabular}
\end{table}

---

\subsection{Intuitive Summary}

\begin{quote}
The Bellman operator \( \mathcal{T}^* \) ``pulls'' any Q-function toward the optimal fixed point \( Q^* \).
Fitted Q-Iteration iteratively applies this operator via function approximation.
In the tabular or linear case, this process is a contraction and converges to \( Q^* \).
In the nonlinear case (e.g., deep neural networks), the projection breaks contraction,
and convergence is no longer guaranteed.
\end{quote}

\begin{table}[h!]
\centering
\caption{Comparison among Q-Iteration, Q-Learning, and Online Q-Iteration}
\resizebox{\textwidth}{!}{
\begin{tabular}{l|l|l|l}
\toprule
\textbf{Aspect} &
\textbf{Q-Iteration} &
\textbf{Q-Learning} &
\textbf{Online Q-Iteration (Function Approximation)} \\
\midrule
\textbf{Mathematical form} &
$Q_{k+1}(s,a) = r(s,a) + \gamma \sum_{s'} P(s'|s,a)\max_{a'} Q_k(s',a')$ &
$Q(s_t,a_t) \leftarrow Q(s_t,a_t) + \alpha [r_t + \gamma \max_{a'} Q(s_{t+1},a') - Q(s_t,a_t)]$ &
$\phi \leftarrow \phi - \alpha \nabla_\phi Q_\phi(s_i,a_i)\,[Q_\phi(s_i,a_i) - (r_i + \gamma \max_{a'} Q_\phi(s'_i,a'))]$ \\
\midrule
\textbf{Core idea} &
Exact Bellman iteration using known model &
Sample-based approximation of Bellman operator &
Gradient descent on Bellman error with function approximation \\
\midrule
\textbf{Environment model $P(s'|s,a)$} &
Required (model-based) &
Not required (model-free) &
Not required (model-free) \\
\midrule
\textbf{Update type} &
Batch / synchronous (all states) &
Incremental / online (per step) &
Online stochastic gradient (per sample) \\
\midrule
\textbf{Function representation} &
Tabular (explicit $Q(s,a)$) &
Tabular (explicit $Q(s,a)$) &
Parameteric $Q_\phi(s,a)$ (e.g., neural network) \\
\midrule
\textbf{Learning signal} &
Exact expectation $\mathbb{E}_{s'}[\cdot]$ &
Single transition sample &
Mini-batch / single-sample gradient step \\
\midrule
\textbf{Policy type} &
Greedy: $\pi(s)=\arg\max_a Q(s,a)$ &
Off-policy (e.g., $\epsilon$-greedy for exploration) &
Off-policy, with many choices of behavior policy \\
\midrule
\textbf{Objective interpretation} &
Find fixed point of Bellman optimality operator $\mathcal{T}^*$ &
Stochastic approximation to fixed point &
Minimize Bellman error: $\frac{1}{2}\mathbb{E}[(Q_\phi - \mathcal{T}^* Q_\phi)^2]$ \\
\midrule
\textbf{Data usage} &
Uses full transition probabilities &
Uses sampled transitions $(s,a,r,s')$ &
Uses replay buffer / online samples $(s,a,r,s')$ \\
\midrule
\textbf{Convergence guarantee} &
Guaranteed (Bellman contraction) &
Guaranteed under conditions (Robbins–Monro) &
Not guaranteed (depends on function class / optimization) \\
\midrule
\textbf{Computation cost} &
High (full state sweep) &
Low (one-step updates) &
High per update (backpropagation, nonlinear fit) \\
\midrule
\textbf{Example algorithms} &
Value Iteration, Dynamic Programming &
Tabular Q-Learning &
DQN, Double DQN, Fitted Q-Iteration \\
\bottomrule
\end{tabular}
}
\end{table}

%==========================
%  表 1:Online vs Offline
%==========================
\begin{table}[h!]
\centering
\caption{Comparison between Online and Offline Reinforcement Learning}
\resizebox{\textwidth}{!}{
\begin{tabular}{l|l|l}
\toprule
\textbf{Aspect} & \textbf{Online RL} & \textbf{Offline RL} \\
\midrule
\textbf{Data source} & Collected in real time during interaction with the environment & Pre-collected static dataset (no further environment interaction) \\
\midrule
\textbf{Update timing} & Update parameters after every step or episode & Train model using fixed data, all updates done offline \\
\midrule
\textbf{Environment access} & Required (agent interacts continuously) & Not required (training purely from data) \\
\midrule
\textbf{Data distribution} & Continuously changes as policy improves & Fixed; determined by behavior policy that collected data \\
\midrule
\textbf{Advantages} & Adaptive, can learn from continuous feedback & Safe, data-efficient, usable in domains where interaction is costly \\
\midrule
\textbf{Challenges} & Instability, high variance, exploration–exploitation tradeoff & Distributional shift, extrapolation error, overestimation \\
\midrule
\textbf{Typical algorithms} & Q-Learning, SARSA, Actor–Critic, DDPG, PPO & Fitted Q-Iteration, DQN (offline), CQL, IQL, Behavior Cloning \\
\bottomrule
\end{tabular}
}
\end{table}

%=============================
%  表 2:On-policy vs Off-policy
%=============================
\begin{table}[h!]
\centering
\caption{Comparison between On-policy and Off-policy Learning}
\resizebox{\textwidth}{!}{
\begin{tabular}{l|l|l}
\toprule
\textbf{Aspect} & \textbf{On-policy} & \textbf{Off-policy} \\
\midrule
\textbf{Definition} & Learns the value of the \emph{same} policy used to generate data & Learns the value of a \emph{different} (target) policy than data-generating policy \\
\midrule
\textbf{Behavior vs. target policy} & Identical ($\pi_b = \pi_t$) & Different ($\pi_b \neq \pi_t$) \\
\midrule
\textbf{Exploration method} & Exploration built into policy (e.g., $\epsilon$-greedy within current policy) & Can reuse off-policy data and experience replay \\
\midrule
\textbf{Advantages} & Stable, lower bias, directly improves current policy & Data-efficient, can leverage past experience and other agents' data \\
\midrule
\textbf{Challenges} & Requires fresh data, less efficient & Risk of distribution mismatch, instability in importance sampling \\
\midrule
\textbf{Typical algorithms} & SARSA, A2C, PPO, Policy Gradient (on-policy) & Q-Learning, DQN, DDPG, SAC, Fitted Q-Iteration \\
\midrule
\textbf{Relation to data} & Must collect new data from current policy & Can learn from arbitrary datasets (even offline) \\
\bottomrule
\end{tabular}
}
\end{table}

\section{Deep RL with Q-Fucntions}
\subsection{Why Q-Learning is \textbf{not} Gradient Descent}

\subsubsection{The Apparent Gradient Descent Form}

In function-approximation Q-learning, the update rule is:

\[
\phi \leftarrow \phi - \alpha \nabla_\phi Q_\phi(s_i, a_i)
\big( Q_\phi(s_i,a_i) - y_i \big),
\quad 
y_i = r(s_i,a_i) + \gamma \max_{a'} Q_\phi(s'_i, a')
\]

This looks like minimizing the following squared error:

\[
L(\phi) = \frac{1}{2} (Q_\phi(s_i,a_i) - y_i)^2.
\]

At first glance, it appears to be a standard gradient descent step.

---

\subsubsection{The Hidden Problem: \textit{Target Depends on } $\phi$}

However, the target \( y_i \) itself depends on the same parameters \( \phi \) through \( Q_\phi(s'_i,a') \).
If we compute the true gradient of \( L(\phi) \):

\[
\nabla_\phi L(\phi)
= (Q_\phi(s_i,a_i) - y_i)
\Big[
\nabla_\phi Q_\phi(s_i,a_i)
- \nabla_\phi \big(r(s_i,a_i) + \gamma \max_{a'} Q_\phi(s'_i,a')\big)
\Big],
\]

we can see that the second term propagates gradient through the target value \(y_i\).
In practice, \textbf{Q-learning ignores this term}:

\[
\text{no gradient through target: } \quad
\nabla_\phi y_i = 0.
\]

Hence, Q-learning performs a *semi-gradient* update, not a true gradient descent step.

---

\subsubsection{Why Ignoring the Gradient is Necessary}

If gradients were allowed to flow through the target \(y_i\), the optimization would become unstable because:
\begin{itemize}
    \item The target \(y_i\) changes as \(Q_\phi\) changes --- a moving target.
    \item The learning objective is not fixed; both sides of the Bellman equation depend on the same parameters.
\end{itemize}

Therefore, Q-learning treats \(y_i\) as a constant:
\[
y_i = \text{stop\_gradient}\big(r_i + \gamma \max_{a'} Q_\phi(s'_i,a')\big),
\]
turning the update into a \textbf{fixed-point iteration} rather than a true gradient-based optimization.

---

\subsubsection{Comparison: Gradient Descent vs Q-Learning Update}

\begin{table}[h!]
\centering
\resizebox{\textwidth}{!}{
\begin{tabular}{l|l|l}
\toprule
\textbf{Aspect} & \textbf{True Gradient Descent} & \textbf{Q-Learning Update} \\
\midrule
Objective & Minimize fixed loss function \(L(\phi)\) & Satisfy Bellman fixed-point equation \\
\midrule
Target \(y\) & Constant (does not depend on $\phi$) & Depends on $\phi$, but gradient stopped \\
\midrule
Gradient flow & Through both prediction and target & Through prediction only \\
\midrule
Type of optimization & Standard supervised learning (convex/nonconvex) & Bootstrapped fixed-point iteration \\
\midrule
Stability & Usually stable (well-defined loss landscape) & Can diverge (moving target problem) \\
\midrule
Interpretation & Minimizing a static objective & Approximating the Bellman operator $\mathcal{T}^*$ \\
\midrule
Example & Linear regression, cross-entropy loss & TD-learning, Q-learning, DQN \\
\bottomrule
\end{tabular}
}
\end{table}

---

\subsubsection{Conceptual Interpretation}

\begin{itemize}
    \item In gradient descent, we descend a \textit{fixed hill}; the loss surface does not move as we update parameters.
    \item In Q-learning, the ``hill'' itself moves, because the target depends on the same parameters — we are chasing a moving target.
\end{itemize}

---

\subsubsection{Connection to DQN}

To mitigate the instability caused by a moving target, DQN introduces a 
\textbf{target network} \(Q_{\phi^-}\) for computing targets:

\[
y_i = r_i + \gamma \max_{a'} Q_{\phi^-}(s'_i, a'),
\quad \text{with } \phi^- \text{ updated slowly (periodically frozen)}.
\]

This makes the target temporarily constant, bringing the update closer to a stable gradient descent process.

---

\subsubsection{Summary}

\begin{quote}
\textbf{Q-learning is not true gradient descent}, because it ignores gradients through its bootstrapped target.
It is a fixed-point iteration aiming to satisfy the Bellman equation, not to minimize a static differentiable loss.
\end{quote}


\subsection{A More General View of Q-Learning and DQN}

\subsubsection{The Three-Process View}

Q-learning with replay buffer and target network can be decomposed into three main processes:

\begin{enumerate}
    \item \textbf{Process 1: Data Collection}  
    Interact with the environment using a policy $\pi(a|s)$ (e.g., $\epsilon$-greedy), 
    collect transitions $(s, a, s', r)$, and store them in the replay buffer $\mathcal{B}$.
    
    \item \textbf{Process 2: Target Update}  
    Update target network parameters periodically or smoothly:
    \[
    \phi' \leftarrow \tau \phi' + (1 - \tau)\phi \quad \text{or} \quad \phi' \leftarrow \phi
    \]
    This provides a stable target for Q-learning.
    
    \item \textbf{Process 3: Q-Function Regression}  
    Sample batches from the replay buffer and perform gradient descent:
    \[
    \phi \leftarrow \phi - \alpha \sum_i 
    \nabla_\phi Q_\phi(s_i,a_i)
    \Big(Q_\phi(s_i,a_i) - [r_i + \gamma \max_{a'} Q_{\phi'}(s'_i,a')]\Big)
    \]
\end{enumerate}

The overall structure is visualized as follows:

\[
\text{Environment} 
\leftrightarrow{\pi(a|s)}
\text{Replay Buffer} 
\longrightarrow 
\text{Current Parameters } (\phi)
\longrightarrow 
\text{Target Parameters } (\phi')
\longrightarrow 
\text{Q-function Regression}
\]

Old transitions are evicted periodically to maintain buffer freshness.

---

\subsubsection{Process Interaction Overview}

\begin{itemize}
    \item \textbf{Process 1 — Data Collection:}  
    Generates new transitions $(s, a, s', r)$ and inserts them into the replay buffer.  
    It determines how the agent explores the environment.

    \item \textbf{Process 2 — Target Update:}  
    Updates target parameters $\phi'$, either hard or soft, defining how frequently the target values change.

    \item \textbf{Process 3 — Q-Function Regression:}  
    Uses sampled batches from $\mathcal{B}$ to update $\phi$ by minimizing the TD error.
\end{itemize}

These processes may run at different speeds or be nested within each other, leading to different algorithmic variants.

---

\subsubsection{Algorithm Comparisons in the Three-Process Framework}

\begin{table}[h!]
\centering
\resizebox{\textwidth}{!}{
\begin{tabular}{l|p{5cm}|p{5cm}|p{5cm}}
\toprule
\textbf{Algorithm} & \textbf{Process 1 (Data Collection)} & \textbf{Process 2 (Target Update)} & \textbf{Process 3 (Q-Function Regression)} \\
\midrule
\textbf{Online Q-Learning} & Collects new transition, immediately discards old ones. & No separate target network (same $\phi$ used). & All processes run simultaneously at same speed. \\
\midrule
\textbf{DQN} & Process 1 and 3 run together (real-time data collection and regression). & Process 2 (target update) runs slowly (periodically or softly). & Regression uses replay buffer to decorrelate samples. \\
\midrule
\textbf{Fitted Q-Iteration} & Data collection (process 1) is outer loop. & Target update (process 2) inside it. & Regression (process 3) runs as an inner optimization loop until convergence. \\
\bottomrule
\end{tabular}
}
\end{table}

---

\subsubsection{Intuitive Summary}

\begin{itemize}
    \item Reinforcement learning can be understood as the interaction of three asynchronous processes:
        \begin{enumerate}
            \item Data generation (environment sampling)
            \item Target computation (stabilization)
            \item Function approximation (learning)
        \end{enumerate}
    \item Different RL algorithms can be viewed as different synchronization choices between these processes.
    \item DQN improves upon online Q-learning by decoupling these processes and allowing stable off-policy updates.
\end{itemize}

---

\subsubsection{Key Insights}

\begin{itemize}
    \item The replay buffer provides a bridge between \textbf{data collection} and \textbf{Q-function learning}.
    \item The target network acts as a temporal buffer between \textbf{learning} and \textbf{bootstrapping}.
    \item By adjusting their relative update speeds, we can control the trade-off between \textit{stability} and \textit{learning speed}.
\end{itemize}

---

\subsubsection{Summary of Process Speed Hierarchies}

\[
\begin{array}{ll}
\textbf{Online Q-learning:} & \text{Process 1, 2, 3 all run at the same rate.} \\
\textbf{DQN:} & \text{Process 1 \& 3 run together; Process 2 (target) is slower.} \\
\textbf{Fitted Q-Iteration:} & \text{Process 3 nested inside Process 2, which is nested inside Process 1.}
\end{array}
\]

\subsection{Q-learning with N-step Returns}

In standard Q-learning, the target value is computed using a single-step reward:
\[
y_t = r_t + \gamma \max_a Q_{\phi'}(s_{t+1}, a)
\]
This approach relies heavily on bootstrapping from the next state, which can introduce high bias and slow propagation of reward information.

To address this, we can use \textbf{N-step returns} to incorporate multiple future rewards:

\[
y_{j,t} = 
\sum_{t' = t}^{t + N - 1} 
\gamma^{t' - t} r_{j,t'} 
+ 
\gamma^N \max_{a_{j,t+N}} 
Q_{\phi'}(s_{j,t+N}, a_{j,t+N})
\]

This estimates the expected return for policy $\pi$ as:
\[
Q^\pi(s_{j,t}, a_{j,t}) \approx y_{j,t}
\]
where the policy $\pi$ is assumed to be:
\[
\pi(a_t|s_t) =
\begin{cases}
1, & \text{if } a_t = \arg\max_a Q_\phi(s_t, a) \\
0, & \text{otherwise.}
\end{cases}
\]

---

\subsubsection*{Advantages}

\begin{itemize}
    \item \textbf{Less biased target values:} Incorporates multiple real rewards, reducing dependence on inaccurate Q-values early in training.
    \item \textbf{Faster learning:} Reward information propagates backward more quickly, improving sample efficiency.
\end{itemize}

---

\subsubsection*{Limitation: Off-policy Bias}

The N-step formulation assumes that all intermediate transitions 
$(s_{t'}, a_{t'})$ for $t' - t < N-1$ 
are generated by the same policy $\pi$.  
This assumption holds for \textbf{on-policy} learning, but breaks for \textbf{off-policy} settings (e.g., when using a replay buffer).

Therefore, the N-step target is:
\begin{itemize}
    \item \textcolor{red}{\textbf{Only theoretically correct when learning on-policy.}}
    \item \textcolor{gray}{Off-policy data can introduce bias because transitions may come from older behavior policies.}
\end{itemize}

For $N = 1$, this issue disappears since only one transition $(s_t, a_t, r_t, s_{t+1})$ is required, which is valid for any policy.

---

\subsubsection*{Practical Fixes for Off-policy Settings}

\begin{enumerate}
    \item \textbf{Ignore the problem} – Simply apply N-step targets without correction.  
    In practice, this often works well for small $N$.
    \item \textbf{Cut the trace} – Dynamically choose $N$ to include only on-policy segments.  
    Effective when most samples are near on-policy and the action space is small.
    \item \textbf{Importance sampling} – Weight each step by 
    $\prod_t \frac{\pi(a_t|s_t)}{\mu(a_t|s_t)}$ 
    to correct off-policy bias.  
    This is theoretically correct but may increase variance.
\end{enumerate}

---

\subsubsection*{Summary}

\begin{table}[h!]
\centering
\renewcommand{\arraystretch}{1.3}
\resizebox{\textwidth}{!}{
\begin{tabular}{l|p{6cm}|p{6cm}}
\toprule
\textbf{Aspect} & \textbf{Advantage} & \textbf{Limitation / Concern} \\
\midrule
Bias & Reduces bias when Q-values are inaccurate & Only unbiased for on-policy data \\
Learning Speed & Propagates reward information faster & May introduce instability off-policy \\
Implementation & Easy to extend from standard Q-learning & Needs multiple consecutive transitions \\
Off-policy Use & Can be corrected via importance sampling & High variance in correction weights \\
\bottomrule
\end{tabular}
}
\end{table}

---

\subsubsection*{Reference}

For a formal treatment, see:
\begin{quote}
Munos et al., \textit{``Safe and Efficient Off-Policy Reinforcement Learning''}, 2016.
\end{quote}
\subsection{Double Deep Q-learning (Double DQN)}

\subsubsection*{Motivation}

Although the classic Deep Q-learning (DQN) algorithm greatly stabilizes Q-learning with replay buffers and target networks, 
it still suffers from \textbf{overestimation bias}.  

In standard DQN, the target is computed as:
\[
y_t = r_t + \gamma \max_{a'} Q_{\phi'}(s_{t+1}, a')
\]
Since both the selection and evaluation of the maximizing action are done using the same Q-network $Q_{\phi'}$, 
any noise or over-optimistic estimate in $Q_{\phi'}$ tends to be \textbf{amplified by the $\max$ operator}.

This phenomenon can systematically overestimate Q-values, leading to suboptimal policies.

---

\subsubsection*{Key Idea}

\textbf{Double DQN} (van Hasselt et al., 2016) addresses this problem by \emph{decoupling action selection and evaluation}.

Instead of using the same network for both roles, Double DQN uses:
\begin{itemize}
    \item The \textbf{online network} $Q_{\phi}$ to \textbf{select} the greedy action.
    \item The \textbf{target network} $Q_{\phi'}$ to \textbf{evaluate} that action.
\end{itemize}

The new target value is defined as:
\[
y_t^{\text{DoubleDQN}} = 
r_t + 
\gamma \,
Q_{\phi'} \bigl( s_{t+1}, 
\arg\max_{a'} Q_{\phi}(s_{t+1}, a') 
\bigr)
\]

That is, the $\arg\max$ is computed by the current (online) parameters $\phi$, 
but the value is obtained from the slower-moving target parameters $\phi'$.

---

\subsubsection*{Algorithm}

\begin{enumerate}
    \item Collect transitions $(s_t, a_t, r_t, s_{t+1})$ into replay buffer $\mathcal{B}$.
    \item Sample a mini-batch $\{ (s_j, a_j, r_j, s'_j) \}$ from $\mathcal{B}$.
    \item Compute targets using:
    \[
    y_j = r_j + \gamma 
    Q_{\phi'}(s'_j, \arg\max_{a'} Q_{\phi}(s'_j, a'))
    \]
    \item Update the online network parameters $\phi$ by minimizing:
    \[
    L(\phi) = 
    \frac{1}{2} \sum_j 
    \bigl(
    Q_{\phi}(s_j, a_j) - y_j
    \bigr)^2
    \]
    \item Periodically update the target network:
    \[
    \phi' \leftarrow \tau \phi' + (1 - \tau)\phi
    \quad \text{(soft update, or hard copy every $N$ steps)}
    \]
\end{enumerate}

---

\subsubsection*{Intuition}

In standard DQN:
\[
\text{same network for both selection and evaluation} 
\Rightarrow 
\text{biased overestimation}
\]

In Double DQN:
\[
\text{two networks:}
\begin{cases}
\text{Selection: } a^* = \arg\max_a Q_{\phi}(s', a) \\
\text{Evaluation: } Q_{\phi'}(s', a^*)
\end{cases}
\Rightarrow 
\text{more accurate targets, reduced bias.}
\]

This results in more stable learning, especially in stochastic environments where rewards and value estimates are noisy.

---

\subsubsection*{Advantages and Disadvantages}

\begin{table}[h!]
\centering
\renewcommand{\arraystretch}{1.3}
\resizebox{\textwidth}{!}{
\begin{tabular}{l|p{6cm}|p{6cm}}
\toprule
\textbf{Aspect} & \textbf{Advantage} & \textbf{Limitation / Concern} \\
\midrule
Estimation Bias & Reduces overestimation in Q-values & May introduce slight underestimation \\
Stability & Improves training stability compared to vanilla DQN & Still sensitive to hyperparameters \\
Computation & Requires only minor changes to DQN (same architecture) & Needs careful synchronization between $\phi$ and $\phi'$ \\
Empirical Results & Performs significantly better on Atari benchmarks & Benefits diminish for very large networks or continuous actions \\
\bottomrule
\end{tabular}
}
\end{table}

---

\subsubsection*{Summary}

\begin{itemize}
    \item Double DQN separates action selection and evaluation, effectively correcting the overestimation bias of standard DQN.
    \item It retains the computational simplicity of DQN while yielding more stable and reliable performance.
    \item It is now considered a standard component in most modern deep reinforcement learning algorithms (e.g., Rainbow DQN, Dueling DQN).
\end{itemize}

---

\subsubsection*{Reference}

\begin{quote}
van Hasselt, H., Guez, A., and Silver, D. (2016).  
\textit{Deep Reinforcement Learning with Double Q-learning}.  
In Proceedings of the 30th AAAI Conference on Artificial Intelligence.
\end{quote}

\end{CJK}
\end{document}