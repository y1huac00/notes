\documentclass[12pt]{article}

% 数学宏包
\usepackage{amsmath, amssymb, amsthm}
\usepackage{CJK}

% 定理环境设置
\theoremstyle{definition} % 定义类(直立字体)
\newtheorem{definition}{Definition}[section]

\theoremstyle{plain} % 定理类(斜体)
\newtheorem{theorem}[definition]{Theorem}
\newtheorem{lemma}[definition]{Lemma}
\newtheorem{corollary}[definition]{Corollary}
\newtheorem{proposition}[definition]{Proposition}

\theoremstyle{remark} % 备注类(直立字体,带斜体标题)
\newtheorem{remark}[definition]{Remark}
\newtheorem{example}[definition]{Example}

% 证明环境
\renewcommand\qedsymbol{$\blacksquare$} % QED 符号改成实心方块

% 页面与字体
\usepackage[a4paper, margin=1in]{geometry}
\usepackage{lmodern} % 更好看的字体

% 数学集合常用符号
\newcommand{\R}{\mathbb{R}}
\newcommand{\Z}{\mathbb{Z}}
\newcommand{\N}{\mathbb{N}}

\begin{document}
\begin{CJK}{UTF8}{gbsn}
\section{PAC Learning Framework}

\begin{definition}[Generalization error]
Given a hypothesis $h \in \mathcal{H}$, a target concept $c \in \mathcal{C}$,  
and an underlying distribution $\mathcal{D}$, the generalization error or risk of $h$ is defined by
\[
R(h) = \Pr_{x \sim \mathcal{D}}[h(x) \neq c(x)] 
     = \mathbb{E}_{x \sim \mathcal{D}} \big[ \mathbf{1}_{h(x) \neq c(x)} \big],
\]
where $\mathbf{1}_\omega$ is the indicator function of the event $\omega$.
\end{definition}

\begin{definition}[Empirical error]
Given a hypothesis $h \in \mathcal{H}$, a target concept $c \in \mathcal{C}$, 
and a sample $S = (x_1, \ldots, x_m)$, the empirical error or empirical risk of $h$ is defined by
\[
\hat{R}_S(h) = \frac{1}{m} \sum_{i=1}^m \mathbf{1}_{h(x_i) \neq c(x_i)}.
\]
\end{definition}

\begin{definition}[PAC-learning]
A concept class $\mathcal{C}$ is said to be PAC-learnable \emph{if there exists an algorithm $A$ and a polynomial function $\mathrm{poly}(\cdot,\cdot,\cdot,\cdot)$ such that for any $\epsilon > 0$ and $\delta > 0$, for all distributions $\mathcal{D}$ on $\mathcal{X}$ and for any target concept $c \in \mathcal{C}$, the following holds for any sample size $m \geq \mathrm{poly}(1/\epsilon, 1/\delta, n, \mathrm{size}(c))$:}
\[
\Pr_{S \sim \mathcal{D}^m}\!\left[ R(h_S) \leq \epsilon \right] \geq 1 - \delta.
\]
If $A$ further runs in $\mathrm{poly}(1/\epsilon, 1/\delta, n, \mathrm{size}(c))$, 
then $\mathcal{C}$ is said to be \emph{efficiently PAC-learnable}.
When such an algorithm $A$ exists, it is called a PAC-learning algorithm for $\mathcal{C}$.
\end{definition}

\begin{theorem}[Learning bound --- finite $\mathcal{H}$, consistent case]
Let $\mathcal{H}$ be a finite set of functions mapping from $X$ to $Y$. 
Let $A$ be an algorithm that for any target concept $c \in \mathcal{H}$ 
and i.i.d.\ sample $S$ returns a consistent hypothesis $h_S$: 
$\hat{R}_S(h_S) = 0$. Then, for any $\epsilon, \delta > 0$, 
the inequality
\[
\Pr_{S \sim \mathcal{D}^m}\!\big[ R(h_S) \leq \epsilon \big] \geq 1 - \delta
\]
holds if
\[
m \geq \frac{1}{\epsilon} \left( \log |\mathcal{H}| + \log \frac{1}{\delta} \right).
\tag{2.8}
\]

This sample complexity result admits the following equivalent statement 
as a generalization bound: for any $\epsilon, \delta > 0$, with probability at least $1-\delta$,
\[
R(h_S) \leq \frac{1}{m} \left( \log |\mathcal{H}| + \log \frac{1}{\delta} \right).
\tag{2.9}
\]
\end{theorem}

\begin{corollary}
Fix $\epsilon > 0$. Then, for any hypothesis $h: X \to \{0,1\}$, 
the following inequalities hold:
\[
\Pr_{S \sim \mathcal{D}^m}\!\left[ \hat{R}_S(h) - R(h) \geq \epsilon \right] 
    \leq \exp(-2m\epsilon^2). \tag{2.14}
\]
\[
\Pr_{S \sim \mathcal{D}^m}\!\left[ \hat{R}_S(h) - R(h) \leq -\epsilon \right] 
    \leq \exp(-2m\epsilon^2). \tag{2.15}
\]

By the union bound, this implies the following two-sided inequality:
\[
\Pr_{S \sim \mathcal{D}^m}\!\left[ \,|\hat{R}_S(h) - R(h)| \geq \epsilon \,\right] 
    \leq 2 \exp(-2m\epsilon^2). \tag{2.16}
\]
\end{corollary}

\begin{corollary}[Generalization bound --- single hypothesis]
Fix a hypothesis $h: X \to \{0,1\}$. 
Then, for any $\delta > 0$, the following inequality holds with probability at least $1 - \delta$:
\[
R(h) \leq \hat{R}_S(h) + \sqrt{\tfrac{\log \tfrac{2}{\delta}}{2m}}.
\tag{2.17}
\]
\end{corollary}

\begin{theorem}[Learning bound --- finite $\mathcal{H}$, inconsistent case]
Let $\mathcal{H}$ be a finite hypothesis set. 
Then, for any $\delta > 0$, with probability at least $1 - \delta$, the following inequality holds:
\[
\forall h \in \mathcal{H}, \quad 
R(h) \leq \hat{R}_S(h) + \sqrt{\tfrac{\log |\mathcal{H}| + \log \tfrac{2}{\delta}}{2m}}.
\tag{2.20}
\]
\end{theorem}

\begin{definition}[Agnostic PAC-learning]
Let $\mathcal{H}$ be a hypothesis set. 
An algorithm $A$ is an agnostic PAC-learning algorithm if there exists a polynomial function $\mathrm{poly}(\cdot,\cdot,\cdot,\cdot)$ such that for any $\epsilon > 0$ and $\delta > 0$, for all distributions $\mathcal{D}$ over $\mathcal{X} \times \mathcal{Y}$, the following holds for any sample size 
$m \geq \mathrm{poly}(1/\epsilon, 1/\delta, n, \mathrm{size}(c))$:
\[
\Pr_{S \sim \mathcal{D}^m}\!\Big[ R(h_S) - \min_{h \in \mathcal{H}} R(h) \leq \epsilon \Big] \geq 1 - \delta.
\tag{2.21}
\]
If $A$ further runs in $\mathrm{poly}(1/\epsilon, 1/\delta, n)$, then it is said to be an efficient agnostic PAC-learning algorithm.
\end{definition}

\begin{definition}[Bayes error]
Given a distribution $\mathcal{D}$ over $\mathcal{X} \times \mathcal{Y}$, 
the Bayes error $R^*$ is defined as the infimum of the errors achieved by measurable functions 
$h: \mathcal{X} \to \mathcal{Y}$:
\[
R^* = \inf_{h \ \text{measurable}} R(h). \tag{2.22}
\]

A hypothesis $h$ with $R(h) = R^*$ is called a \emph{Bayes hypothesis} or \emph{Bayes classifier}.

By definition, in the deterministic case, we have $R^* = 0$, but, in the stochastic case, 
$R^* \neq 0$. Clearly, the Bayes classifier $h_{\text{Bayes}}$ can be defined in terms of the 
conditional probabilities as:
\[
\forall x \in \mathcal{X}, \quad 
h_{\text{Bayes}}(x) = \arg\max_{y \in \{0,1\}} \Pr[y \mid x]. \tag{2.23}
\]

The average error made by the bayes hypothesis on $x\in\mathcal{X}$ is thus $\min\{\Pr[0|x],\Pr[1|x]\}$, 
and this is the minimum possible error.

\end{definition}

\begin{definition}[Noise]
Given a distribution $\mathcal{D}$ over $\mathcal{X} \times \mathcal{Y}$, 
the noise at point $x \in \mathcal{X}$ is defined by
\[
\mathrm{noise}(x) = \min\{\Pr[1 \mid x], \Pr[0 \mid x]\}. \tag{2.24}
\]

The average noise or the noise associated to $\mathcal{D}$ is 
$\mathbb{E}[\mathrm{noise}(x)]$.

Thus, the average noise is precisely the Bayes error: 
\[
\mathrm{noise} = \mathbb{E}[\mathrm{noise}(x)] = R^*.
\]

The noise is a characteristic of the learning task indicative of its level of difficulty. 
A point $x \in \mathcal{X}$, for which $\mathrm{noise}(x)$ is close to $1/2$, is sometimes referred to as \emph{noisy}, 
and is of course a challenge for accurate prediction.
\end{definition}


\section{Rademacher Complexity and VC-Dimension}

\begin{definition}[Empirical Rademacher complexity]
Let $\mathcal{G}$ be a family of functions mapping from $\mathcal{Z}$ to $[a,b]$ and 
$S = (z_1, \ldots, z_m)$ a fixed sample of size $m$ with elements in $\mathcal{Z}$. 
Then, the empirical Rademacher complexity of $\mathcal{G}$ with respect to the sample $S$ 
is defined as:
\[
\hat{\mathfrak{R}}_S(\mathcal{G}) = \mathbb{E}_{\sigma} \left[ 
\sup_{g \in \mathcal{G}} \frac{1}{m} \sum_{i=1}^m \sigma_i g(z_i)
\right], \tag{3.1}
\]
where $\sigma = (\sigma_1, \ldots, \sigma_m)^\top$, with $\sigma_i$ independent uniform 
random variables taking values in $\{-1,+1\}$. The random variables $\sigma_i$ are 
called \emph{Rademacher variables}.

Let $\mathbf{g}_S$ denote the vector of values taken by function $g$ over the sample $S$: 
$\mathbf{g}_S = (g(z_1), \ldots, g(z_m))^\top$. Then, the empirical Rademacher complexity 
can be rewritten as
\[
\hat{\mathfrak{R}}_S(\mathcal{G}) 
= \mathbb{E}_{\sigma} \left[ \sup_{g \in \mathcal{G}} \frac{\sigma \cdot \mathbf{g}_S}{m} \right].
\]

The inner product $\sigma \cdot \mathbf{g}_S$ measures the correlation of $\mathbf{g}_S$ 
with the vector of random noise $\sigma$. The supremum 
$\sup_{g \in \mathcal{G}} \frac{\sigma \cdot \mathbf{g}_S}{m}$ is a measure of how well 
the function class $\mathcal{G}$ correlates with $\sigma$ over the sample $S$. 
Thus, the empirical Rademacher complexity measures on average how well the function 
class $\mathcal{G}$ correlates with random noise on $S$. This describes the richness of 
the family $\mathcal{G}$: richer or more complex families $\mathcal{G}$ can generate more 
vectors $\mathbf{g}_S$ and thus better correlate with random noise, on average.
\end{definition}

\begin{definition}[Rademacher complexity]
Let $\mathcal{D}$ denote the distribution according to which samples are drawn. 
For any integer $m \geq 1$, the Rademacher complexity of $\mathcal{G}$ is the expectation of the 
empirical Rademacher complexity over all samples of size $m$ drawn according to $\mathcal{D}$:
\[
\mathfrak{R}_m(\mathcal{G}) 
= \mathbb{E}_{S \sim \mathcal{D}^m} \big[ \hat{\mathfrak{R}}_S(\mathcal{G}) \big].
\tag{3.2}
\]
\end{definition}

\begin{theorem}
Let $\mathcal{G}$ be a family of functions mapping from $\mathcal{Z}$ to $[0,1]$. 
Then, for any $\delta > 0$, with probability at least $1 - \delta$ over the draw of 
an i.i.d.\ sample $S$ of size $m$, each of the following holds for all $g \in \mathcal{G}$:
\[
\mathbb{E}[g(z)] \;\leq\; \frac{1}{m}\sum_{i=1}^m g(z_i) + 2 \mathfrak{R}_m(\mathcal{G}) 
+ \sqrt{\frac{\log (1/\delta)}{2m}}, \tag{3.3}
\]
and
\[
\mathbb{E}[g(z)] \;\leq\; \frac{1}{m}\sum_{i=1}^m g(z_i) + 2 \hat{\mathfrak{R}}_S(\mathcal{G}) 
+ 3\sqrt{\frac{\log (2/\delta)}{2m}}. \tag{3.4}
\]
\end{theorem}

\begin{theorem}[Hoeffding's inequality]
Let $X_1, \ldots, X_m$ be independent random variables with 
$X_i$ taking values in $[a_i, b_i]$ for all $i \in [m]$. 
Then, for any $\epsilon > 0$, the following inequalities hold for 
\[
S_m = \sum_{i=1}^m X_i:
\]
\[
\Pr\!\big[ S_m - \mathbb{E}[S_m] \geq \epsilon \big] 
\;\leq\; \exp\!\left(-\frac{2\epsilon^2}{\sum_{i=1}^m (b_i - a_i)^2}\right), \tag{D.4}
\]
\[
\Pr\!\big[ S_m - \mathbb{E}[S_m] \leq -\epsilon \big] 
\;\leq\; \exp\!\left(-\frac{2\epsilon^2}{\sum_{i=1}^m (b_i - a_i)^2}\right). \tag{D.5}
\]
\end{theorem}

\begin{theorem}[McDiarmid's inequality]
Let $X_1, \ldots, X_m \in \mathcal{X}^m$ be a set of $m \geq 1$ independent random variables 
and assume that there exist $c_1, \ldots, c_m > 0$ such that 
$f: \mathcal{X}^m \to \mathbb{R}$ satisfies the following conditions:
\[
\big| f(x_1, \ldots, x_i, \ldots, x_m) - f(x_1, \ldots, x_i', \ldots, x_m) \big| \leq c_i,
\quad \forall i \in [m], \;\; \forall x_1, \ldots, x_m, x_i' \in \mathcal{X}. \tag{D.15}
\]

Let $f(S)$ denote $f(X_1, \ldots, X_m)$. Then, for all $\epsilon > 0$, the following 
inequalities hold:
\[
\Pr\!\big[f(S) - \mathbb{E}[f(S)] \geq \epsilon\big] 
\;\leq\; \exp\!\left( \frac{-2\epsilon^2}{\sum_{i=1}^m c_i^2} \right), \tag{D.16}
\]
\[
\Pr\!\big[f(S) - \mathbb{E}[f(S)] \leq -\epsilon\big] 
\;\leq\; \exp\!\left( \frac{-2\epsilon^2}{\sum_{i=1}^m c_i^2} \right). \tag{D.17}
\]
\end{theorem}

\begin{lemma}
Let $\mathcal{H}$ be a family of functions taking values in $\{-1,+1\}$ and let 
$\mathcal{G}$ be the family of loss functions associated to $\mathcal{H}$ for the zero-one loss: 
\[
\mathcal{G} = \{(x,y) \mapsto \mathbf{1}_{h(x) \neq y} : h \in \mathcal{H}\}.
\]  
For any sample $S = ((x_1,y_1), \ldots, (x_m,y_m))$ of elements in 
$\mathcal{X} \times \{-1,+1\}$, let $S_X$ denote its projection over $\mathcal{X}$: 
$S_X = (x_1, \ldots, x_m)$.  
Then, the following relation holds between the empirical Rademacher complexities of 
$\mathcal{G}$ and $\mathcal{H}$:
\[
\hat{\mathfrak{R}}_S(\mathcal{G}) = \tfrac{1}{2}\,\hat{\mathfrak{R}}_{S_X}(\mathcal{H}). \tag{3.16}
\]
\end{lemma}

\begin{theorem}[Rademacher complexity bounds -- binary classification]
Let $\mathcal{H}$ be a family of functions taking values in $\{-1,+1\}$ and let 
$\mathcal{D}$ be the distribution over the input space $\mathcal{X}$. 
Then, for any $\delta > 0$, with probability at least $1 - \delta$ over a sample 
$S$ of size $m$ drawn according to $\mathcal{D}$, each of the following holds for any 
$h \in \mathcal{H}$:
\[
R(h) \;\leq\; \hat{R}_S(h) + \mathfrak{R}_m(\mathcal{H}) 
+ \sqrt{\frac{\log \tfrac{1}{\delta}}{2m}}, \tag{3.17}
\]
and
\[
R(h) \;\leq\; \hat{R}_S(h) + \hat{\mathfrak{R}}_S(\mathcal{H}) 
+ 3\sqrt{\frac{\log \tfrac{2}{\delta}}{2m}}. \tag{3.18}
\]
\end{theorem}

\begin{definition}[Growth function]
The growth function $\Pi_{\mathcal{H}} : \mathbb{N} \to \mathbb{N}$ for a hypothesis set $\mathcal{H}$ is defined by:
\[
\forall m \in \mathbb{N}, \quad 
\Pi_{\mathcal{H}}(m) = 
\max_{\{x_1,\ldots,x_m\} \subseteq \mathcal{X}}
\left| \left\{ (h(x_1), \ldots, h(x_m)) : h \in \mathcal{H} \right\} \right|.
\tag{3.19}
\]

In other words, $\Pi_{\mathcal{H}}(m)$ is the maximum number of distinct ways in which $m$ points can be classified using hypotheses in $\mathcal{H}$. 
Each one of these distinct classifications is called a \emph{dichotomy} and, thus, the growth function counts the number of dichotomies that are realized by the hypothesis. 
This provides another measure of the richness of the hypothesis set $\mathcal{H}$. 
However, unlike the Rademacher complexity, this measure does not depend on the distribution; it is purely combinatorial.
\end{definition}

\begin{theorem}[Massart's lemma]
Let $\mathcal{A} \subseteq \mathbb{R}^m$ be a finite set, with 
\[
r = \max_{\mathbf{x} \in \mathcal{A}} \|\mathbf{x}\|_2,
\] 
then the following holds:
\[
\mathbb{E}_{\sigma} \left[ \frac{1}{m} \sup_{\mathbf{x} \in \mathcal{A}} \sum_{i=1}^m \sigma_i x_i \right] 
\leq \frac{r \sqrt{2 \log |\mathcal{A}|}}{m},
\tag{3.20}
\]
where $\sigma_i$s are independent uniform random variables taking values in $\{-1,+1\}$ and $x_1, \ldots, x_m$ are the components of vector $\mathbf{x}$.
\end{theorem}

\begin{corollary}
Let $\mathcal{G}$ be a family of functions taking values in $\{-1,+1\}$. 
Then the following holds:
\[
\mathfrak{R}_m(\mathcal{G}) \leq \sqrt{\frac{2 \log \Pi_{\mathcal{G}}(m)}{m}}.
\tag{3.21}
\]
\end{corollary}

\begin{corollary}[Growth function generalization bound]
Let $\mathcal{H}$ be a family of functions taking values in $\{-1,+1\}$. 
Then, for any $\delta > 0$, with probability at least $1-\delta$, 
for any $h \in \mathcal{H}$,
\[
R(h) \leq \hat{R}_S(h) 
+ \sqrt{\frac{2 \log \Pi_{\mathcal{H}}(m)}{m}} 
+ \sqrt{\frac{\log \tfrac{1}{\delta}}{2m}}.
\tag{3.22}
\]
\end{corollary}

\begin{definition}[shattering]
A set $S$ of at least one point is said to be shattered by a hypothesis set $\mathcal{H}$ when $\mathcal{H}$ realizes all possible dichotomies of $S$, i.e. $\Pi_{\mathcal{H}}(m)=2^m$.    
\end{definition}

\begin{definition}[VC-dimension]
The VC-dimension of a hypothesis set $\mathcal{H}$ is the size of the largest set that can be shattered by $\mathcal{H}$:
\[
\mathrm{VCdim}(\mathcal{H}) = \max \{ m : \Pi_{\mathcal{H}}(m) = 2^m \}. \tag{3.24}
\]
\end{definition}

Note that, by definition, if $\mathrm{VCdim}(\mathcal{H}) = d$, there exists a set of size $d$ that can be shattered. However, this does not imply that all sets of size $d$ or less are shattered and, in fact, this is typically not the case.

\begin{theorem}[Radon's theorem]
Any set $X$ of $d+2$ points in $\mathbb{R}^d$ can be partitioned 
into two subsets $X_1$ and $X_2$ such that the convex hulls of $X_1$ 
and $X_2$ intersect.
\end{theorem}

\begin{theorem}[Sauer's lemma]
Let $\mathcal{H}$ be a hypothesis set with $\mathrm{VCdim}(\mathcal{H}) = d$. 
Then, for all $m \in \mathbb{N}$, the following inequality holds:
\[
\Pi_{\mathcal{H}}(m) \leq \sum_{i=0}^{d} \binom{m}{i}.
\]
\end{theorem}

\begin{corollary}
Let $\mathcal{H}$ be a hypothesis set with $\mathrm{VCdim}(\mathcal{H}) = d$. 
Then for all $m \geq d$,
\[
\Pi_{\mathcal{H}}(m) \leq \left( \frac{em}{d} \right)^d = O(m^d).
\]
\end{corollary}

\begin{corollary}[VC-dimension generalization bounds]
Let $\mathcal{H}$ be a family of functions taking values in $\{-1,+1\}$ with VC-dimension $d$. 
Then, for any $\delta > 0$, with probability at least $1-\delta$, the following holds for all $h \in \mathcal{H}$:
\[
R(h) \leq \hat{R}_S(h) + \sqrt{\frac{2d \log \frac{em}{d}}{m}} + \sqrt{\frac{\log \frac{1}{\delta}}{2m}}.
\tag{3.29}
\]
Thus, the form of this generalization bound is
\[
R(h) \leq \hat{R}_S(h) + O\!\left(\sqrt{\frac{\log(m/d)}{(m/d)}}\right).
\tag{3.30}
\]
\end{corollary}


\begin{theorem}[Lower bound, realizable case]
Let $\mathcal{H}$ be a hypothesis set with VC-dimension $d > 1$. 
Then, for any $m \geq 1$ and any learning algorithm $A$, 
there exist a distribution $\mathcal{D}$ over $\mathcal{X}$ and a target function $f \in \mathcal{H}$ such that
\[
\mathbb{P}_{S \sim \mathcal{D}^m} \left[ R_{\mathcal{D}}(h_S, f) > \frac{d-1}{32m} \right] \geq \frac{1}{100}.
\tag{3.31}
\]
\end{theorem}

\begin{lemma}[Lemma 3.21]
Let $\alpha$ be a uniformly distributed random variable taking values in $\{\alpha_{-}, \alpha_{+}\}$, where 
$\alpha_{-} = \tfrac{1}{2} - \tfrac{\epsilon}{2}$ and $\alpha_{+} = \tfrac{1}{2} + \tfrac{\epsilon}{2}$, 
and let $S$ be a sample of $m \geq 1$ random variables $X_1, \ldots, X_m$ taking values in $\{0,1\}$ 
and drawn i.i.d. according to the distribution $\mathcal{D}_\alpha$ defined by 
$\mathbb{P}_{\mathcal{D}_\alpha}[X=1] = \alpha$. 
Let $h$ be a function from $X^m$ to $\{\alpha_{-}, \alpha_{+}\}$. Then, the following holds:
\[
\mathbb{E}_\alpha \Bigg[ \mathbb{P}_{S \sim \mathcal{D}_\alpha^m} \big[ h(S) \neq \alpha \big] \Bigg] 
\geq \Phi(2 \lceil m/2 \rceil, \epsilon),
\]
where
\[
\Phi(m, \epsilon) = \tfrac{1}{4}\Bigg( 1 - \sqrt{1 - \exp\Big(- \tfrac{m\epsilon^2}{1-\epsilon^2}\Big)} \Bigg), 
\quad \text{for all $m$ and $\epsilon$.}
\]
\end{lemma}

\begin{lemma}[Lemma 3.22]
Let $Z$ be a random variable taking values in $[0,1]$. Then, for any $\gamma \in [0,1)$,
\[
\mathbb{P}[Z > \gamma] \;\geq\; \frac{\mathbb{E}[Z] - \gamma}{1-\gamma} \;>\; \mathbb{E}[Z] - \gamma.
\]
\end{lemma}

\begin{theorem}[Theorem 3.23, Lower bound, non-realizable case]
Let $\mathcal{H}$ be a hypothesis set with VC-dimension $d > 1$. 
Then, for any $m \geq 1$ and any learning algorithm $A$, 
there exists a distribution $\mathcal{D}$ over $\mathcal{X} \times \{0,1\}$ such that:
\[
\mathbb{P}_{S \sim \mathcal{D}^m} \Bigg[ R_{\mathcal{D}}(h_S) - \inf_{h \in \mathcal{H}} R_{\mathcal{D}}(h) 
> \sqrt{\tfrac{d}{320m}} \Bigg] \;\geq\; \tfrac{1}{64}.
\]
Equivalently, for any learning algorithm, the sample complexity verifies
\[
m \;\geq\; \frac{d}{320\epsilon^2}.
\]
\end{theorem}

\section{Model Selection}
\subsection{Convex surrogate losses}
Problem: Solving ERM optimization problem is NP-hard since the 0-1 loss function is not convex.
Solution: Use alternative convex surrogate losses that upper bounds the 0-1 loss. 
For $h:\mathcal{X} \to \R$, define following binary classifier:
\[
f_h(x) = 
\begin{cases}
+1 & \text{if } h(x) \geq 0, \\
-1 & \text{if } h(x) < 0.
\end{cases}
\]
We have
\[
1_{f_h(x)\neq y} = 1_{y h(x)<0} + 1_{h(x)=0 \wedge y=-1} \leq 1_{y h(x)\leq 0}.
\]
For any \( x \in \mathcal{X} \), let 
\(\eta(x) = \mathbb{P}[y=+1 \mid x]\) 
and let \(\mathcal{D}_\mathcal{X}\) denote the marginal distribution over \(\mathcal{X}\). 
Then, for any \( h \), we can write
\[
\begin{aligned}
R(h) 
&= \mathbb{E}_{x \sim \mathcal{D}_\mathcal{X}} 
   \big[ \eta(x) 1_{h(x)<0} + (1-\eta(x)) 1_{h(x)>0} + (1-\eta(x))1_{h(x)=0} \big] \\
&= \mathbb{E}_{x \sim \mathcal{D}_\mathcal{X}} 
   \big[ \eta(x) 1_{h(x)<0} + (1-\eta(x)) 1_{h(x)\geq 0} \big].
\end{aligned}
\]

In view of that, the Bayes classifier can be defined as assigning label \(+1\) to \(x\) 
when \(\eta(x) \geq \tfrac{1}{2}\), and \(-1\) otherwise. 
It can therefore be induced by the function \( h^* \) defined by
\[
h^*(x) = \eta(x) - \tfrac{1}{2}. \tag{4.9}
\]

We will refer to \( h^* : \mathcal{X} \to \mathbb{R} \) as the 
\textit{Bayes scoring function} and will denote by \( R^* \) the Bayes error: 
\[
R^* = R(h^*).
\]

\begin{lemma}[Lemma 4.5]
The excess error of any hypothesis $h : \mathcal{X} \to \mathbb{R}$ 
can be expressed as follows in terms of $\eta$ and the Bayes scoring function $h^*$:
\[
R(h) - R^* \;=\; 2 \, \mathbb{E}_{x \sim \mathcal{D}_{\mathcal{X}}} 
\Big[\, |h^*(x)| \, 1_{\,h(x) h^*(x) \leq 0} \,\Big].
\]
\end{lemma}
$|h^*(x)|$接近0时说明贝叶斯分类器也不确定$x$的分类,因此这时候如果$h$分类与$h^*$不一致,对excess error的贡献小,反之大。因为在二分类中,错误概率和正确概率是互补的,差值计算时会出现一个 2 倍的加权项。

\end{CJK}
\end{document}
